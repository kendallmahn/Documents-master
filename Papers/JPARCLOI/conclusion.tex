\section{Conclusion}

The proposed \nuprismlite detector has the potential to address the remaining systematic uncertainties that are not well constrained by ND280. In particular, this detector can constrain the relationship between measured lepton kinematics and incident neutrino energy without relying solely on rapidly-evolving neutrino interaction models. Since \nuprismlite is a water Cherenkov detector, the neutral current backgrounds with large systematic uncertainties at Super-K, particularly NC$\pi^+$ and NC$\pi^0$, can be measured directly with a nearly identical neutrino energy spectrum. The ability to produce nearly monoenergetic neutrino beams also provides the first ever ability to measure neutral current cross sections as a function of neutrino energy. Finally, \nuprismlite provides a mechanism to separate the many single-ring e-like event types to simultaneously constrain $\nu_e$ cross sections, neutral current background, and sterile neutrino oscillations.

The main long-baseline oscillation analysis presented in this note was a $\nu_\mu$ disappearance measurement, since the effects of various cross section models on this measurement had already been well studied, which provided a useful basis for comparison. However, it is also expected that \nuprismlite will provide a significant improvement to the ultimate T2K constraint on $\delta_{CP}$ by constraining neutral current backgrounds and electron-neutrino cross sections. Initial studies have also been presented that demonstrate the impact \nuprismlite can have on both $\nu_e$ appearance measurements and anti-neutrino oscillation measurements. Other planned improvements to the analysis include a realistic detector simulation and event reconstruction. Thanks to the work done on event simulation and reconstruction in Hyper-K, these tools already exist and can be quickly incorporated into the current analysis to perform more detailed studies of event pileup and detector performance for various detector configurations and PMT sizes and coverage.

Cost estimates for \nuprismlite are still preliminary, but initial quotes have been obtained for the most expensive components of the project: the civil construction and PMTs. Initial quotes have been received for these two items, which provides an initial cost estimate for the total project of US\$16.3 million.  There are still uncertainties associated with this cost estimate that will be reduced before the experiment proposal is submitted.  More details regarding the project cost can be found in the appendix. Once funded, \nuprismlite is expected to take less than 3 years to construct, based on the experience from the T2K 2~km detector.


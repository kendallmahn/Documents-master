\section{Measurements with CC $\nu_{\mu}$ candidates \label{sec:nuprism_numu}}

% Need to somehow also talk about NC here too.

As discussed in the previous section, $\nu$PRISM can reconstruct large, pure samples of charged current
events with a muon candidate and no other particles above Cherenkov threshold.  For each event, the
off-axis angle $\theta_{OA}$ is reconstructed and this observable implies additional information about the
distribution of possible neutrino energies based on the prior flux model.  Data binned in $\theta_{OA}$ and the
usual muon observables of momentum (or kinetic energy) and scattering angle are used to constrain the
cross section model.  In the typical approach, the data may be unfolded to find the cross-section in the
true variable of neutrino energy, muon momentuma and muon scattering angle.  However, unfolding relies 
on regularization to deal with the ill-defined nature of the problem, and may not perform well when 
unfolding for $\theta_{OA}$ to the neutrino energy.

We take an alternative approach of using the off-axis neutrino spectra impinging on $\nu$PRISM as a set of basis
functions.  We then write a spectrum of interest as a linear combination of the off-axis spectra:
\begin{equation}
F_{\nu_{\mu}}(E_{\nu}) = \sum_{i} c_{i}\phi^{i}_{\nu_{\mu}}(E_{\nu}).
\end{equation}
Here $F_{\nu_{\mu}}(E_{\nu})$ is an arbitrary function of interest, $\phi^{i}_{\nu_{\mu}}(E_{\nu})$ is the predicted spectrum
in the $i^{th}$ off-axis angle bin and $c_{i}$ are simply coefficients.  The challenge is to find a set of $c_{i}$
that approximatly satisfy the equation while minimizing the statistical and systematic uncertainties that are propagated through the
linear combination.  Once the $c_{i}$ are found, we can write a similar equation for the observables:
\begin{equation}
N(p_{\mu},\theta_{\mu}|F_{\nu_{\mu}}(E_{\nu})) = \sum_{i} c_{i}N^{i}(p_{\mu},\theta_{\mu}).
\end{equation}
In this way, the final state particle multiplicities and kinematics can be measured for an given choice of 
of the input neutrino energy spectrum.  

The consequence of the nuPRISM flux combinations are that we can generate any neutrino flux shape within the kinematics specified by the detector position and beam line. Two immediate applications are evident: pseudo-monoenergetic neutrino beams for measurements of neutrino interactions on water, discussed in Section~\ref{sec:mono}, and the neutrino flux assuming a set of oscillation parameters, used  for long-baseline experiments such as T2K or Hyper-Kamiokande (Section~\ref{sec:oscd}) or atmospheric neutrino oscillation experiments.

\subsection{Pseudo-monoenergetic neutrino flux measurements}
\label{sec:mono}

% Neutrino vs. electron scattering.
%% Relevant variables of eA

While neutrinos are a unique probe of the axial structure of the nucleus, it has been challenging to use them. The canonical probe of nuclear structure is to scatter an electron of a known energy off a nuclear target. A comparison between the incident and recoil electron determines the energy transfer ($\omega$) and momentum transfer ($q^2$). Electron scattering experiments determine 

% Challenges of current measurements
%% CC1pi+ disagreement in deuterium data, CCQE axial form factor -> MEC?, NC/CC disagreements.
%% Flux averaged results, complicated by nuclear effects.

% What is possible with nuPRISM:
%% examples of mono energetic beams, ranges.
%% example eA plots?

% measurements of CCQE, CC1pi+
%% reference what has been done historically.
% nue/numu cross section ratio
% Unique measurements of NC processes.
%% Lack of information about NCpi+, NC1gamma. List of processes expected based on historic measurements **refs 1kton



\subsection{Oscillated }
\label{sec:oscd}

